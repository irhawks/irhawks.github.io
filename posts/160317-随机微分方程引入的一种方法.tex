---
title : 随机微分方程的一种引入方法
date  : 2016-03-17
tags  : 随机微分方程, 随机过程, 
---

我们该怎样引入随机微分方程呢?大概是这样的,首先考虑常的微分方程ODE为$x'=V(x)$, $x\in \Omega\subseteq\mathbb{R}^n$,然后此方程刻划物理现象的时候作为物体运动的轨迹,其解$x(t), t\geq 0$是关于$t$的光滑的曲线。但是在许多应用中,从实验所观测到的轨迹不如确定性方程所描述的那样光滑。因此有必要改正ODE为SDE。

也就是变成微分
\begin{equation}
d X(t) = V(X(t)) d t + G(X(t))\xi(t)
\end{equation}
的形式。其中的$G(X)$是$n\times m$的矩阵。$\xi$是$m$维的白噪声。但是问题在于,在数学上如何严格定义白噪音。作为一个随机过程,其解又是什么意思?

直观的解释是使用连续随机过程来处理随机微分方程。每个$X$表示是一个随机变量,而$X(t)$整体表示的是求解一族随机过程本身。因此也就变成了求随机变量分布或者密度与时间的参数的关系。比如线性单自由度体系的运动方程
\begin{equation}
mX''(t) + cX'(t) + kX(t) = Y(t), X(0) = X_0, X'(0) = X'_0
\end{equation}
中,引入$X_1(t) = X(t)$,就写成使用状态变量描述动态系统的方法。这种方法是系统工程与现代控制理论的重要的手段。在振动工程中有许多方便的应用。

\section{公式化定义}

考虑具有随机初始条件的简单随机微分方程
\begin{equation}
X'(t) = f(X(t),t) , t\in T; X(t_0) = X_0
\end{equation}
式中的$f$是均方连续且均方有界的函数。$X_0$是已知的随机变量。给定一个随机过程$X(t)$,如果它满足
\begin{enumerate}
\item $X(t)$在$T$上均方连续
\item $X(t_0) = X_0$
\item $f(X(t),t)$是$X(t)$在$T$上的均方导数,则称$X(t)$是方程的一个均方解。易证方程与积分方程
\begin{equation}
X(t) = X_0 + \int_{t_0}^t f(X(s),s)\d s
\end{equation}
是等价的。其中的积分是均方积分。这样对随机微分方程的求解经常可以从随机积分方程入手。
\end{enumerate}

具有随机初始条件的齐次线性常微分方和可以描述为:
\begin{equation}
X'(t) = A(t) X(t), t\in T, X(t_0) = X_0
\end{equation}
式中的$A(t)$是$n\times n$的确定性的实矩阵,且各元素均在$t$上连续。$X_0$则是已知的某个随机变量,比如服从正态分布的随机变量。其对应的具有确定初始条件的方程是
\begin{equation}
x'(t) = A(t) x(t), t\in T, x(t_0) = x_0
\end{equation}

由常微分方程理论我们就知道这样的常微分方程具有唯一解$x(t) = H(t,t_0)x_0$。其中$H$是$n\times n$的矩阵,它与系统矩阵$A(t)$对应确定性方程的基本解矩阵。当$A(t)=A$是常数矩阵的时候,$H(t,t_0)$可以显式地表示成矩阵指数

$$H(t,t_0) = \mathrm{e}^{(t-t_0)A}$$

类似地,我们考察随机微分方程具有和常微分方程类似形式的解,那么应该有$X(t) = H(t,t_0), X(0)=X_0$。
